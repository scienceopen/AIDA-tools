\documentclass[amstex,psfig]{article}

\begin{document}

\section{General overview and introduction}
\label{sec:gener-overv-intr}

\begin{itemize}
\item
  Loading images - giving the image as well as information about
  date and time and place of observation.
  \begin{itemize}
  \item This involves reading of images in all formats possible not
    only standard/ALIS fits images.
  \item Further it is preferable if all sorts of header syntaxes are
    understood and correctly parsed, this is not a advanced but
    difficult task.
  \item A good feature would be if the program made as good a guess as
    possible of the optical parameters based on the information in the
    FITS header.
  \end{itemize}
\item Zooming in on individual stars and determining their shape and
  intensity.
  \begin{itemize}
  \item This is something that works well enough in the current
    implementation.
  \end{itemize}
\item Semi/Fully automated search ad identification of stars.
  \begin{itemize}
  \item The semi-automated identification should preferably be updated
    so that it is possible to stop and resume identification at will.
  \item Fully automated identification is desirable but the current
    understanding is that it will not be achieved without much work
    and then with doubtful quality.
  \end{itemize}
\item Manual/Automatic search/optimisation of optical parameters.
  \begin{itemize}
  \item This should be streamlined. So that it will always produce.
  \end{itemize}
\item Merging a star-field over the image.
  \begin{itemize}
  \item This works well. 
  \item One thing would be that dragging and pushing the star-chart
    would alter the optical parameters.
  \end{itemize}
\item Saving and plotting of the result should be made easier so that
  it is straightforward to understand.
\item The current implementation clutters the matlab workspace with
  variables --- this must be addressed
\end{itemize}
\end{document}

%%% Local Variables: 
%%% mode: latex
%%% TeX-master: t
%%% End: 
